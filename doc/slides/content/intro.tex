\section{Introduction}

\begin{frame}{Je me présente}
  \begin{block}{Bérenger Ossété Gombé}
    \begin{itemize}
    \item Bac Scientifique (2013).
    \item Maîtrise en informatique (2017).
    \item Reconversion web chez OpenClassRooms (janvier 2022).
    \end{itemize}
  \end{block}
\end{frame}

\begin{frame}{Contexte du projet}
  \begin{block}{Projet n°8, Parcours Python}
    \begin{itemize}
    \item Développer une application web en utilisant Django
    \item Utiliser le rendu côté serveur dans Django
    \end{itemize}
  \end{block}
\end{frame}

\begin{frame}{Contexte fictif}
  \begin{center}
    \includegraphics[scale=0.2]{img/logo.png}
  \end{center}
  
  \begin{block}{LITReview}
    \begin{itemize}
    \item Équipe
      \begin{itemize}
      \item Alix, UX \textit{designer}.
      \item Sam, le directeur technique.
      \item Nous sommes \textit{lead} développeur Python.
      \end{itemize}
    \item Objectif: $\rightarrow$ Développement d'une application web.
    \end{itemize}    
  \end{block}
\end{frame}

\begin{frame}{L'application}
  \begin{block}{Deux types d'utilisateurs}
    \begin{itemize}
    \item Les utilisateurs qui \textbf{demandent des critiques} de documents.
    \item Les utilisateurs qui \textbf{recherchent des documents} guidés par les critiques.
    \end{itemize}
  \end{block}
\end{frame}
